\documentclass{article}

\usepackage[intlimits]{amsmath}
\usepackage{amsfonts}
\usepackage{amscd}
\usepackage{amssymb}
\usepackage{natbib}


\ifx\pdfoutput\undefined
\usepackage[dvips]{graphicx}
\else
\usepackage[pdftex]{graphicx}
\pdfcompresslevel=9
\usepackage{epstopdf}
\fi

\usepackage{rotating}
\usepackage{lscape}

\title{Interactive Programs \\
	{\normalsize A Fortran 90 library for multitaper spectrum analysis}}
\date{}
\begin{document}
\DeclareGraphicsExtensions{.pdf,.gif,.jpg, .eps}

\maketitle

\section{Introduction}

The present documentation is for the Interactive Programs for multitaper spectral analysis as presented in 
\begin{quotation}
Prieto, G. A., R. L. Parker and F. L. Vernon (2008). A Fortran 90 library for multitaper spectrum analysis. Submitted to Computers and Geosciences, February 2008.
\end{quotation}
The library contains a comprehensive and easy to use collection of subroutines for univariate and multivariate spectral analysis. For those not familiar with Fortran and for easy experimentation with the various parameters and methods in spectral analysis I created these interactive programs. The user need not exit the programs to experiment with different parameter settings or different data files. The programs make on-the-fly plots using {\it plotxy} if requested by the user or saves the results in user-defined data files.

All programs run a command-line interpreter (based on Bob Parker's programs). The user enters command words from a catalog of accepted ones. Options not required may be omitted and defaults will be chosen by the program as necessary.  

\section{Requirements}
The first requirement is to have a Fortran 90 or 95 compiler available in your machine. If plotting is needed, the program {\it PLOTXY} needs to be installed in your machine (the source file is included in this release) and ghostview (gv) is also needed for viewing the plots. 

The examples shown in the \texttt{examples} folder are reproducing the figures accompanying the paper above. 

\section{Interactive Programs}
There are a number of interactive programs to perform univariate and multivariate spectral analysis. The available programs are:
\begin{description}
\item[\texttt{mtpsd} ] Multitaper spectral analysis
\item[\texttt{mtpad} ] Multitaper spectral analysis with zero padding
\item[\texttt{coherence} ] Coherence estimation
\item[\texttt{trfunction} ] Transfer function estimation
\item[\texttt{deconv} ] Multitaper Deconvolution
\item[\texttt{dual\_freq} ] Dual frequency analysis (dual-frequency coherence)
\item[\texttt{wigner} ] Multitaper Wigner-Ville spectrum
\end{description}

Using the program \texttt{mtpsd} an example of a simple program that produces a spectral estimate and plots it is
\begin{quote}
\begin{verbatim}
   file ../data/PASC.dat
   plot
   exec
   quit
\end{verbatim}
\end{quote}
Data values are in a single column in the ASCII file PASC.dat; they are sampled evenly in time and the program assumes 1 sps (sample-per-second). \texttt{mtpsd} uses as default the Thomson multitaper methods, but the user could easily use the sine multitaper by typing instead
\begin{quote}
\begin{verbatim}
   file ../data/PASC.dat
   method sine
   plot
   exec
   quit
\end{verbatim}
\end{quote}

Below is an example of a more elaborate program. The spectrum is first computed with a single prolate taper, then a standard multitaper, quadratic multitaper and the sine multitaper methods are used on the same data file. There is no pre-whitening performed before computing the spectrum. 
\begin{verbatim}
command                          Explanation
----------------------------------------------------------------

file ../data/PASC.dat       Open file with data
method thomson              Method to use
plot 0                      Plot only results for spectra
logxy loglog                Log-Log plot
tbnw 1.5                    Time bandwidth for one taper
kspec 1                     Use only one taper
interval 1.                 Sampling interval
save noise_st.dat           Save file with results
exec                        execute command 
 
method thomson
plot 0
tbnw 4.5                    Increase time bandwidth product
kspec 7                     Use 7 tapers, not one
save noise_mt.dat           Save file with MT results
exec                        Execute command with new parameters
 
method qispec               Use another method, the quadratic mt
plot 0
tbnw 4.5                    Same time-bandwidth as previous
kspec 7                     Same number of tapers as previous
save noise_qi.dat           Save new spectral estimate
exec                        Execute command with new parameters
 
method sine                 Use the sine multitaper method
                            Since no parameters are specified 
                            (ntap or ntimes), the default is used
                            An adaptive spectrum is calculated
plot 0
save noise_sin.dat          Save new sine multitaper estimate
exec                        Execute, with sine MT method
 
quit                        Quit program
\end{verbatim}

In the folder \texttt{examples} a number of example command-line scripts were created to reproduce the figures in the paper. Additional editing of figures was performed for publication quality. 

\section{General features}
I will not describe each individual program, since it is intended to be self-explanatory. When a program is started, the user is asked to provide commands from a catalog of accepted command words. The command word can be abbreviated to its 4 first letters. 

The general procedure is to enter a collection of commands describing a particular calculation, then enter the command \texttt{execute} or \texttt{exec} for short. When the calculation is completed, control is returned to the user for new command words to be input. All commands issued to this point remain in force, but can be replaced is subsequent calculations as required. 

The programs print out the current list of commands before beginning the calculation. The user can print the list of commands at any time by the \texttt{review} command. The most recent commands in the list are enforced. To turn off a command simply use the \texttt{clear} command. 

To know which commands are accepted by the program type \texttt{?} and a brief version of the command catalog is printed on the screen. 

The most important command is of course the \texttt{file} command, where the user defines the filename to be read. If multivariate spectral analysis is needed, two different files can be read or a single one, but data must be in different columns. 
\vspace{0.5cm}

\begin{quote}
Germ\'an A. Prieto \\
Stanford, California \\
\today
\end{quote}






\end{document} 